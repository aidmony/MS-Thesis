%%%%%%%%%%%%%%%%%%%%%%%%%%%%%%%%%%%%%%%%%%%%%%%%%%%%%%%%%%%%%%%%%%%
%                                                                 %
%                            APPENDICES                           %
%                                                                 %
%%%%%%%%%%%%%%%%%%%%%%%%%%%%%%%%%%%%%%%%%%%%%%%%%%%%%%%%%%%%%%%%%%%

\appendix % This command is used only once!
%\addcontentsline{toc}{chapter}{APPENDICES} %toc entry or:
\addtocontents{toc}{\parindent0pt\vskip12pt APPENDICES} %toc entry, no page #
\chapter{}
Follow Gansner's description, we denote the d-dimensional location matrix as $X$~\cite{Gansner:05}. Let node $i$'s location be $X_{i}$. The stress function in literature is defined as
\begin{equation}\label{stress1}
stress(X)=\sum_{i<j} w_{ij} (||X_{i}-X_{j}|| - d_{ij})^{2}\,,
\end{equation}
where $w_{ij}$ are weights and $d_{ij}$ denote the ideal distance between $i$ and $j$. The MDS is an optimization problem to minimize the total stress. Equation~\ref{stress1} can be decomposed and we have
\begin{equation}\label{stress2}
stress(X)=\sum_{i<j} w_{ij}d_{ij}^{2} + \sum_{i<j}w_{ij}||X_{i}-X_{j}||^{2}- 2\sum_{i<j}w_{ij}d_{ij}||X_{i}-X_{j}||.
\end{equation}
The first term of~\ref{stress2} is a constant. The second term, $\sum_{i<j}w_{ij}||X_{i}-X_{j}||^{2}$, is a quadratic sum, and hence can be written as
\[ \sum_{i<j}w_{ij}||X_{i}-X_{j}||^{2} = Tr(X^{T}L^{w}X)\,, \]
where $L^{w}$ is a weighted matrix, defined as
\[
  L_{i,j}^{w}=\left\{ 
  \begin{array}{l l}
    -w_{ij} \quad & i \neq j\\
    \sum_{k \neq i}w_{ik} \quad &i=j 
  \end{array} \right. \text{ .}
\]
The third term, $\sum_{i<j}w_{ij}d_{ij}||X_{i}-X_{j}||$, need extra attention. By Cauchy-Schwartz inequality, 
\[
||X_{i}-X_{j}|| ||Z_{i}-Z_{j}|| \geq (X_{i}-X_{j})^{T} (X_{i} - Z_{j})
\]
with the equality when $Z=X$. Hence, we have the following bound of the third term.
\[
\sum_{i<j}w_{ij}d_{ij}||X_{i}-X_{j}|| \geq \sum_{i<j}w_{ij}d_{ij} (X_{i}-X_{j})^{T} (X_{i} - Z_{j}) / ||Z_{i}-Z_{j}||.
\]
Equivalently, the right side of the inequality can be rewritten in the trace form.
\[
\sum_{i<j}w_{ij}d_{ij} (X_{i}-X_{j})^{T} (X_{i} - Z_{j}) / ||Z_{i}-Z_{j}||=Tr(X^{T}L^{Z}X),
\]
where $L^{Z}$ is defined as
\[
  L_{i,j}^{Z}=\left\{ 
  \begin{array}{l l}
    -w_{ij}d_{ij}/||Z_{i}-Z_{j}|| \quad & i \neq j\\
    -\sum_{j \neq i}L_{i,j}^{Z} \quad &i=j 
  \end{array} \right. \text{ .}
\]
From the above all, we can bound the stress function confidently by function $F^{Z}(X)$, defined as
\begin{equation}
F^{Z}(X)=\sum_{i<j}w_{ij}d_{ij}^{2}+Tr(X^{T}L^{w}X)-2Tr(X^{T}L^{Z}Z)
\end{equation}
with equality when $Z=X$.
Notice $^{Z}(X)$ is of quadratic form, and we can easily find its minima by differentiate it by X and solve
\[
L^{w}X=L^{Z}Z.
\]
This leads to the following iterative optimization process. Given the location matrix at time $t$, $X(t)$, we want to compute $X(t+1)$ so that $stress(X(t+1)) < stress(X(t))$. We take $X(t+1)$ as the minimizer of $F^{X(t)}(X)$ by solving the above equation iteratively. At this point, if $X(t+1)=X(t)$, the process is terminated. Otherwise, we get
\[
stress(X(t+1)) \leq F^{X(t)}(X(t+1)) < F^{x(t)}(X(t)) =stress(X(t)).
\]
To summarize, the majorization process involves solving the following equation iteratively
\begin{equation}
L^{w}X(t+1)=L^{X(t)}X(t),
\end{equation}
where $L^{w}$ is constant throughout the process, whereas matrix $L^{X(t)}$ need to be computed at each iteration.
While stress majorization is an effective and accurate optimization strategy, it is not feasible in very large networks. Solving (14) in each iteration requires the computation of a fully dense matrix, which in total requires $O(n^3)$ time and $O(n^{2})$ space. For network of node number $> 10^{5}$, direct implementation of stress majorization becomes impossible.
