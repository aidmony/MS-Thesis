%%%%%%%%%%%%%%%%%%%%%%%%%%%%%%%%%%%%%%%%%%%%%%%%%%%%%%%%%%%%%%%%%%% 
%                                                                 %
%                            CHAPTER SEVEN                          %
%                                                                 %
%%%%%%%%%%%%%%%%%%%%%%%%%%%%%%%%%%%%%%%%%%%%%%%%%%%%%%%%%%%%%%%%%%% 
 
\chapter{CONCLUSIONS AND FUTURE WORK}
In this thesis, we introduced a general model for structural balance
theory that can handle relation strengths and generalizes the
classical balance theory. We showed that our notion of balance can be
mapped to triangular inequality over metric distances and the issue of
convergence can be modeled as the metric multidimensional scaling
problem for which stress majorization provides exact solutions. We
have shown that our theory can be used to effectively solve the edge
sign prediction problem and its performance matches and exceeds state
of the art for this problem. This is due to the fact that positive and
negative edges are mapped to a continuous range of strengths based on
the constraints provided by the other nodes. However, in contrast with
previous work, our method is aware of global constraints based on
balance which results in better results overall. Furthermore, the
solutions provided by our method can also be used to solve the harder
edge prediction problem.

We are investigating various avenues of future work. Inspired by the
work in~\cite{Khoury:12}, we have developed an approximation algorithm
of stress majorization in the context of social networks. We would like to implement and test how well the approximation is. Furthermore, we are currently testing how the inclusion of relation
strength improves performance by considering other actions of the
users that imply the existence of a social tie. Our method also has
applications to many related problems like clustering and link
prediction, which we are currently investigating. We are studying the
various properties of our theory in general networks and how it can be
extended to an asymmetric interpretation of links.  We note that our
method not only allows us to make predictions, but also study the
characteristics existing networks and compare various
characteristics. Example measures would be the ratio between positive
and negative distances or between largest and smallest
distances. These measures can help us develop new insights into the
nature of adversarial relationships in different networks.