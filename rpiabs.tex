%%%%%%%%%%%%%%%%%%%%%%%%%%%%%%%%%%%%%%%%%%%%%%%%%%%%%%%%%%%%%%%%%%% 
%                                                                 %
%                            ABSTRACT                             %
%                                                                 %
%%%%%%%%%%%%%%%%%%%%%%%%%%%%%%%%%%%%%%%%%%%%%%%%%%%%%%%%%%%%%%%%%%% 
 
\specialhead{ABSTRACT}

Structural balance theory studies the influence between positive and negative relationships in a social network. Researchers regard structural balance as a sound and universal concept in social networks. We first observe the fact that current structural balance theory does not distinguish between varying strengths of relationships, which is shown to be of good significance in many social interactions. We address this problem by analyzing the social and psychological source of imbalance, and establish an extended theory that defines balance in networks with varying relationship strengths.  We show how balance is reasoned when strengths of relationships are measured by either a totally ordered set, or by numerical values. Our extended balance theory is shown to be a generalization of the current structural balance theory.

The convergence model studies how an imbalanced network evolves towards new balance. The assumption behind our model of convergence is that in resolving tensions within imbalanced relationships, people tend to avoid the effort involved in changing relationships if possible. The introduction of extended balance theory allows us to formulate the convergence problem of a social network as a Metric Multidimensional Scaling (MDS) optimization problem. 

Modeling the dynamic nature of social networks, the convergence model inherits a predictive power on unknown relationships. We show that a list of famous social network phenomena, such as ``If two people have more common friends, then they are more likely to become friends", hold under the convergence model. Moreover, we use the convergence model to study the {\it edge sign prediction problem}. In a social network with positive and negative signed edges, the task of sign prediction is to infer the signs of a small fraction of ``hidden" edges based on the information from the rest of the network. We show why and how the convergence model can be applied in prediction tasks over real online social network datasets. Stress majorization technique is used to compute the convergence, and our method consistently matches and outperforms the state of art.